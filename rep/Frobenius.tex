\chapter{Вычисление канонических форм XS-схем}\label{Protsec}

\subsection{Intro}

Будем рассматривать только регулярные $XS$-схемы. Как было доказано в ?? в регулярных схемах матрицу $B$ можно привести к нормальной форме Фробениуса, причем состоять она будет только из одной клетки. Одноклеточная каноническая матрица $B$ имеет вид:      \[\left(\begin{array}{cccccc} {0} & {0} & {\cdots } & {0} & {0} & {b_{1} } \\ {1} & {0} & {\cdots } & {0} & {0} & {b_{2} } \\ {0} & {1} & {\cdots } & {0} & {0} & {b_{3} } \\ {\cdots } & {\cdots } & {\cdots } & {\cdots } & {\cdots } & {\cdots } \\ {0} & {0} & {\cdots } & {1} & {0} & {b_{n-1} } \\ {0} & {0} & {\cdots } & {0} & {1} & {b_{n} } \end{array}\right).\]
Ее характеристический многочлен:
\begin{equation}\label{Frobenius.CharPoly}
f_B\left({\lambda }\right)={\lambda }^n+b_n{\lambda }^{n-1}+\dots +b_1.
\end{equation}

\subsection{Приведение матрицы $B$ к нормальной форме Фробениуса}

Пусть матрица $B'$ нормальная форма Фробениуса матрицы $B$. Тогда существует обратимая матрица $P$ такая, что $P^{-1}BP=B'$. Пусть $s_1, ... , s_n$ столбцы матрицы $P$, а $b'=(b'_{1}, ... , b'_{n})^{T}$ последний столбец $B'$, тогда из равенства $BP=B'P$ следует, что:
\[s_2 = Bs_1\]
\[s_3 = Bs_2\]
$$\vdots $$
\[s_n = Bs_{n-1}\]
\[Pb' = Bs_n.\]
В итоге получаем, что для матрицы $P$ должны выполняться равенства: $s_3=B^2s_1$, $s_4=B^3s_1$, ..., $s_n=B^{n-1}s_1$ и $Pb' = Bs_n$. Последнее равенство в итоге может быть записано в виде
\begin{equation}\label{Frobenius.Eq2}
(s_1,Bs_1,...,B^{n-1}s_1)b'=B^ns_1 \Leftrightarrow (b'_1+b'_2B+ ... +b'_nB{n-1})s_1=B^ns_1.
\end{equation}
Заметим, что так как мы используем только преобразования подобия, то характеристические многочлены матриц $B$ и $B'$ равны, а по теореме Гамильтона-Кэли матрица $B$ будет являться корнем своего характеристического многочлена $f_{B'}\left({\lambda }\right)={\lambda }^n+b'_n{\lambda }^{n-1}+\dots +b'_1$. Значит $f_{B'}\left({B}\right)=0={B}^n+b'_n{B }^{n-1}+\dots +b'_1 \Leftrightarrow B^n=b'_n{B }^{n-1}+\dots +b'_1$. Подставляя это в равенство 
~\ref{Frobenius.Eq2} получим верное тождество. В итоге имеем, что для того чтобы привести матрицу $B$ к нормальной форме Фробениуса, достаточно, чтобы матрица $P$ имела вид $(s_1,Bs_1,...,B^{n-1}s_1)$ и была обратима.

